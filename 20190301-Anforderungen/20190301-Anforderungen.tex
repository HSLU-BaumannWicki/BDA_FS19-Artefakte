\documentclass[parskip=full, a4paper]{scrreprt}

%%% PACKAGES %%%

% add unicode support and use german as language
\usepackage[utf8]{inputenc}
\usepackage[ngerman]{babel}

% Use Helvetica as font
\usepackage[scaled]{helvet}
\renewcommand\familydefault{\sfdefault}
\usepackage[T1]{fontenc}

% Better tables
\usepackage{tabularx}

% Better enumerisation env
\usepackage{enumitem}

% Use graphics
\usepackage{graphicx}

% Have subfigures and captions
\usepackage{subcaption}

% Be able to include PDFs in the file
\usepackage{pdfpages}

% Have custom abstract heading
\usepackage{abstract}

% Need a list of equation
\usepackage{tocloft}
\usepackage{ragged2e}

% Better equation environment
\usepackage{amsmath}

% Symbols for most SI units
\usepackage{siunitx}

\usepackage{csquotes}

% Clickable Links to Websites and chapters
\usepackage{hyperref}

% Change page rotation
\usepackage{pdflscape}

% Symbols like checkmark
\usepackage{amssymb}
\usepackage{pifont}

\usepackage[absolute]{textpos}

% Glossary, hyperref, babel, polyglossia, inputenc, fontenc must be loaded before this package if they are used
\usepackage{glossaries}
% Redefine the quote charachter as we are using ngerman
\GlsSetQuote{+}
% Define the usage of an acronym, Abbreviation (Abbr.), next usage: The Abbr. of ...
\setacronymstyle{long-short}

% Bibliography & citing
\usepackage[
backend=biber,
style=apa,
bibstyle=apa,
citestyle=apa,
sortlocale=de_DE
]{biblatex}
\addbibresource{Referenzen.bib}
\DeclareLanguageMapping{ngerman}{ngerman-apa}

%%% TOC Header
\addto\captionsngerman{
	\renewcommand{\contentsname}{Traktanden}
}

%%% Not clearpage before chapter
\usepackage{etoolbox}
\makeatletter
\patchcmd{\scr@startchapter}{\if@openright\cleardoublepage\else\clearpage\fi}{}{}{}
\makeatother

%%% Fallback DocumentVersion if Builded local
\providecommand{\docversion}{0.0-localBuild}

%%% DOCUMENT %%%
\begin{document}

\chapter{Projektanforderungen v.\docversion}
Zeil: Mittels einer Machbarkeitsstudie und einem Proof of Concept soll untersucht werden ob es möglich ist bis zu 120 RFID Tags in einem Behälter mit der Dimension 600x400x320mm zu identifizieren.

\begin{itemize}
	\item Es sollen mindestens zwei Lösungskonzepte für eine als Auswahl der Machbarkeitsstudie entwickelt werden.
	\item Die Lösungskonzepte müssen auf deren technische Realisierbarkeit untersucht werden.
	\item Es muss mindestens ein entwickeltes Konzept für die Machbarkeitsstudie verwendet werden.
	\item Die Machbarkeitsstudie muss eine Kostenrechnung für die Lösungsansätze beinhalten.
	\item Es soll eine MVP\footnote[1]{Minimal Viable Product \url{https://de.wikipedia.org/wiki/Minimum_Viable_Product}} entwickelt werden, welches vom Kunde verwendet werden kann.
\end{itemize}

\subsubsection{Anforderungen an Lösungsansätze, Proof of Concept und MVP}

\begin{itemize}
	\item Die Lösungskonzepte müssen mit dem Lagersystem kommunizieren können
	\item Die Lösungskonzepte müssen die RFID Tags in weniger als 1 Sekunden identifizieren können.
	\item Die Lösungskonzepte müssen für das bestehende Hochregallager der Speicherbibliothek verwendbar sein.
	
	\item Das Proof of Concept muss technisch aufzeigen, wie viele RFID Tags in einer Sekunde identifiziert werden können.
	
	\item Das MVP soll mit einer Oracle Datenbank kommunizieren können.
	\item Das MVP ist in der Lage die Buch ID eines Exemplares über RFID auszulesen.
	\item Das MVP soll erkennen, wenn eine Box ein Exemplar (eines, welches mit RFID ausgestattet ist und technisch auch Lesbar ist) enthält, welches nicht dieser Box zugehörig ist.
	\item Das MVP soll jede Unstimmigkeit (Exemplar, welches nicht zu diesem Behälter gehört) in einem Logdokument persistieren.
	\item Das MVP soll in der Lage sein, dem Endbenutzer in graphischer Form durch einen Consolenoutput mitzuteilen, welcher Behälter eine Unstimmigkeit enthält.
	\item Das MVP soll die unter Laborbedingungen erhaltenen Resultate unter Realbedingungen verifizieren.
\end{itemize}

\vfill

Mit dieser Unterschrift wird bestätigt, dass die Anforderungen gelesen, verstanden und akzeptiert wurden.

\begin{tabularx}{\textwidth}{lX}
	& \\
	Unterschrift: & \\
	\cline{2-2}
	& \\[0.5cm]
	Ort / Datum: &  \\
	\cline{2-2}
\end{tabularx}
\end{document}